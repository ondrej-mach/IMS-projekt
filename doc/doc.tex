\documentclass[12pt,a4paper]{article}

\usepackage[czech]{babel}
\usepackage{csquotes}
\usepackage{graphicx}
\usepackage{textcomp}
% rovnice zarovnávat doleva
\usepackage[fleqn]{amsmath}

% neodsazovat nové odstavce
\setlength{\parindent}{0pt}


\begin{document}

%%%%%%%%%%%%%%%% TITLE PAGE %%%%%%%%%%%%%%%%
\begin{titlepage}
\begin{center}
\includegraphics[width=0.5\linewidth]{img/logo.pdf}
\vspace{3cm}

\LARGE\uppercase{Modelování a simulace 2021/2022}
\vspace{1cm}

\LARGE\textbf{Simulační studie na technologii Vehicle-to-grid}

\vspace*{\fill}
\large{Ondřej Mach (xmacho12)}

\large{Rostislav Lán (xlanro00)}

\end{center}
\end{titlepage}


%%%%%%%%%%%%%%%% TABLE OF CONTENTS %%%%%%%%%%%%%%%%
\pagenumbering{arabic}
\setcounter{page}{1}
\tableofcontents
\clearpage

%%%%%%%%%%%%%%%% THE ACTUAL DOCUMENT %%%%%%%%%%%%%%%%

\section{Úvod}
Tato práce pokládá otázku, zda je technologie Vehicle-to-grid ekonomicky rentabilní.
Přínos této technologie je posouzen podle simulačního modelu, který modeluje výrobu a spotřebu elektrické energie v rámci domu.
Technologie je poměrně nová a studie na toto téma dochází k různým závěrům, což bylo motivací pro vytvoření tohoto projektu.

\subsection{Technologie Vehicle-to-grid}

\subsection{Předchozí studie}

\section{Simulační model}

\section{Závěr}

\end{document}


















































